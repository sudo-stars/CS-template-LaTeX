\documentclass[11pt,oneside,a4paper]{article}

\usepackage{enumitem}
\usepackage[utf8]{inputenc}
\usepackage{amsmath, amssymb, amsthm}
\usepackage[english]{babel}
\usepackage{graphicx, float}
\graphicspath{{images/}}
\usepackage[table,xcdraw]{xcolor} 
\usepackage[letterpaper, top=1in, left=1in, right=1in, bottom=0.8in, heightrounded]{geometry}
\usepackage{fancyhdr}
\pagestyle{fancy}      
\fancyhf{}             
\fancyhead[LO,L]{WS 25/26}
\fancyhead[CO,C]{Group XX}
\fancyhead[RO,R]{\today} 
\fancyfoot[LO,L]{}
\fancyfoot[CO,C]{\thepage}
\fancyfoot[RO,R]{}
\renewcommand{\headrulewidth}{0.4pt}
\renewcommand{\footrulewidth}{0.4pt}

%-------------------------------------------------------%

\title{\textbf{CS Assignment Template}}
\author{%
  \today \\[1em]
  \begin{tabular}{c|c|c}
      {Name1} & {Name2} & {Name3} \\
      Matrikelnr1 & Matrikelnr2 & Matrikelnr3 \\
  \end{tabular}
}
\date{}

%-------------------------------------------------------%
\begin{document}
\maketitle
\thispagestyle{fancy}

%-------------------------------------------------------%
\section{Aufgabe 1}
\subsection{a)}
Image insert Example
\begin{figure}[h]
    \centering
    \includegraphics[width=1\linewidth]{images/name.jpg}
\end{figure}

%-------------------------------------------------------%
\newpage
\section{Aufgabe 2}
\subsection{a)}
Hoare-Kalkül Verifikation Example
\begin{align*}
    & & \langle a.length-i=m \wedge i<a.length \rangle \\
    & & \langle a.length-(i+1)<m \rangle \\
    & \qquad \mathbf{if} (\text{ s } == \text{ a[i] }) \{ \\
    & & \langle a.length-(i+1)<m \wedge (\text{ s }= \text{ a[i] })\rangle \\
    & \qquad \qquad \text{res = true;} \\
    & & \langle a.length-(i+1)<m \rangle \\
    & \qquad \} \\
    & & \langle a.length-(i+1)<m \rangle \\
    & \quad \text{s = s} + a[i]; \\
    & & \langle a.length-(i+1)<m \rangle \\
    & \quad i=i+1 \\
    & & \langle a.length-i<m \rangle \\
\end{align*}

%-------------------------------------------------------%
\newpage
\section{Aufgabe 7}
\subsection{(Programmierung) }
Coding Example
%-------------Insert code in minted-------------------%
\begin{minted}[fontsize=\small, breaklines, frame=single]{java}
public class Statistics{

    private int[] werte;
    private int counter;

    //Konstruktor
    public Statistics(){
        this.werte = new int[100];
        this.counter = 0;
    }
\end{minted}


%-------------------------------------------------------%
\newpage %Math Example Induktion
\section{Aufgabe 3}
\subsection{(Verallgemeinerte Bernoulli-Ungleichung) }
\textbf{Gegeben:}
$n \in \mathbb{N}$, $a_1,...,a_k \geq -1$
\[
\prod_{k=1}^{n} (1 + a_k) \geq 1 + \sum_{k=1}^{n} a_k
\]
\textbf{Fragestellung:} 
Unter welchem Vorraussetzungen gilt die folgende strikte Ungleichung? \\
\[
\prod_{k=1}^{n} (1 + a_k) > 1 + \sum_{k=1}^{n} a_k
\]

\textbf{Vorüberlegung:} \\
Wir suchen Vorraussetzungen, die die "Gleichheit" eliminieren, so dass nur noch die strikte "grö{\ss}er als" Ungleichung gilt. Wann sind die Terme also gleich? Beginnen wir mit dem kleinsten $n \in \mathbb{N}$:
\begin{align*} n = 1 \\
    &\prod_{k=1}^{1} (1 + a_k) > 1 + \sum_{k=1}^{1} a_k\\
    &=1 + a_1 = 1 + a_1\\
\end{align*}
Die erste Vorraussetzung ist also $n \geq 2$ \\

\textbf{Beweis per Induktionsverfahren:} \\
\\
\textit{IA:} Sei $n=2$
\begin{align*}
    &\prod_{k=1}^{2} (1 + a_k) \\
    &=(1+a_1)(1+a_2)\\
    &=(1+a)^2\\
    &=1+2a+a^2\\
\end{align*}
(...)

\textit{IV:} (...)


\textit{IS:} Sei $n=n+1$.\\
\begin{align*} 
    &\prod_{k=1}^{n+1} (1 + a_k) > 1 + \sum_{k=1}^{n+1} a_k\\
    &=(1+a)^{n+1} > 1 + (n+1) \cdot a \\
\end{align*}
(...)
\qed \\
\textbf{Fazit:} \\
\\
Die strikte Ungleichung gilt unter folgenden Vorraussetzungen:
\begin{enumerate}
    \item $n \in \mathbb{N} \geq 2$ \\
    \item $a \neq 0$
\end{enumerate}


\end{document}
%-------------------------------------------------------%
